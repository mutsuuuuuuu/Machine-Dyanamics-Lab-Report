The primary source of uncertainty for both experiments majorly comes from systematic errors such as human reaction, inaccurate apparatus calibration and incorrect measurement reading because these systematic errors cannot be eliminated and corrected with repeated experiment rerun. On the other hand, the effect of random errors in these experiments are negligible because these unlike systematic errors, these errors can be quickly eliminated by taking multiple readings where in this case to oscillate the object three times. In addition to systematic errors above, a non-symmetrical axis between the object's centre of mass and the suspension will cause inaccurate natural oscillation time period for the second experiment.\\

The uncertainty for the first experiment is approximately $9.1\%$ where as for the second experiment, the uncertainty is much lower uncertainty compare to the uncertainty of the first experiment with $1.9\%$. This shows that the moment of inertia calculated using the trifilar method is more accurate compare to the compound pendulum method in terms of uncertainty calculation. On the down side, the trifilar suspension experiment method is more difficult to set up and to carry because the object's centre of mass must be on a symmetry axis of the suspension in order to get an accurate final result whereas the compound pendulum method is much more easier to set up and carry. \\

The trifilar suspension method could be useful if the object's centre of mass is exactly at half the length of the object or at least the location of centre of mass is known and the surface of the object is flat. This condition will make it easier to put the object symmetrically to the suspension axis which at the end will yield more accurate moment of inertia where as in compound pendulum method, we do not need to worry about the object centre of mass. On the other hand, compound pendulum method could be useful if the oscillation amplitude is small because the equation become less valid as the oscillation amplitude increases therefore could affect the final calculation result. 