The graph on the figure 6 has a $R^2$ value of $0.7829$ which mean $70\%$ of the experimental data fits the linear regression line. On the other hand, the graph on figure 7 has a $R^2$ value of approximately $0.99$ which mean $99\%$ of the experimental data for vee belt resembles the linear regression line from origin. The source of error for this experiment comes from systematic errors that happen during experiment such as the weight balance is not "zero" properly, the angle of lap is not exactly what it meant to be. Moreover, random errors also happen during the experiment and it is something that we cannot control. These errors such as the humidity of the room might coat the surface of the drum with small amount of water therefore causing the coefficient of friction of the drum's surface to either decrease or increase. \\

The experimental results above show limiting slip conditions for each belt types (coil and vee-belt) that were used in the experiment. The limiting slip condition values showed the maximum power that can be transmitted into the belt without slipping. The theoretical relationship can be proven by substituing the value of $T_2$, $\mu$, and $\theta$ into the equation $T_1 = e^{\theta \mu}T_2$ for coil belt and substituting the value of  $T_2$, $\mu$, $\theta$ and $\alpha$ into  $T_1 = e^{\theta \mu '}T_2$ and $\mu = \mu ' \csc\left(\frac{\alpha}{2}\right)$.